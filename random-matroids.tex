\chapter{Random matroids}\label{random-matroids}

In his 1974 paper, Knuth~\cite{knuth-1975} describes a mechanism for generating random matroids in terms of their closed sets. The definition of a matroid in terms of its closed sets can be found in part~\ref{background/matroids}.

Let $\mathfrak{M} = (E, \mathcal{F})$ be a matroid, where $E$ is the ground set and $\mathcal{F}$ the family of closed sets of $\mathfrak{M}$. Since $\mathcal{F}$ is a subset of the $2^E$, the number of possible such families (and thus different matroids over $E$) is bounded by $2^{2^{|E|}}$. Therefore it is infeasible to look at all possible families of closed subsets over E. Knuth uses a bottom-up approach instead, constructing $\mathcal{F}$ by starting with the smallest closed sets and incrementally constructing the larger ones. This mechanism is presented in Algorithm~\ref{alg:knuth}.

\begin{algorithm}[float*=ht,width=\textwidth]{\pr{Knuth-Matroid}(E, \mathrm{X})}{knuth}

  \textbf{Input:} The ground set of elements $E$, and a list of enlargements $\mathrm{X}$.

  \textbf{Output:} The list of closed sets of the resulting matroid grouped by rank, $\mathrm{F} = [F_0, \ldots, F_r]$, where $F_i$ is the set of closed sets of rank $i$.

  \begin{pseudo}[label=\small\arabic*, indent-mark, line-height=1.1]
    $r = 0, \mathrm{F} = [\{ \emptyset \}, \{ \emptyset \}]$ \\
    while true  \\+
    $\mathrm{F}[r+1] = \{ A \cup \{a\} : A \in \mathrm{F}[r], a \in E \setminus A \}$\\
    \pr{Enlarge}(\mathrm{F}[r+1], \mathrm{X}[r+1]) \\
    \pr{Superpose}(\mathrm{F}[r+1], \mathrm{F}[r]) \\

    if $E \not \in F[r+1]$ \\+
    $r \leftarrow r+1$ \\-
    else \\+
    return $\mathrm{F}$

  \end{pseudo}

\end{algorithm}

\pr{Knuth-Matroid} accepts two arguments, the ground set of elements for the matroid, and a sequence $\mathrm{X}$ of ``enlargements,'' where $X_{r+1}$ is the set of enlargements two apply at step $r+1$. The algorithm relies on two subroutines, \pr{Enlarge} and \pr{Superpose}, which will be explained shortly. The algorithm incrementally constructs a list of sets of subsets of $E$, each element a family $F_i \subseteq 2^E$ of closed sets of rank $i$. When the algorithm terminates, the list $\mathrm{F} = [F_0, \ldots, F_r]$ is returned, where $F_i$ is the family of closed sets of rank $i$. In the paper, Knuth shows that $\bigcup_{i=0}^r F_i = \mathcal{F}$, and so the algorithm outputs valid matroids.

The algorithm initializes with $r = 0$. The only closed set of rank 0 is the empty set, so $F_r = F_0 = \{ \emptyset \}$. Inside the loop, on line 3, we produce $F_{r+1}$ from $F_r$ by generating all ``covers'' of the sets in $F_r$, meaning all sets in $F_r$ with one more element added from $E$. Had we only done this for $|E|$ iterations, we would have ended up with the uniform matroid. Not terribly interesting! The \pr{Enlarge} subroutine is what allows us to generate arbitrary matroids, by selectively enlarging the family of closed sets of a given rank. The subroutine enlarges $F_{r+1}$ by simply adding the sets supplied to the algorithm in $X_{r+1}$.

The second subroutine, \pr{Superpose}, does the following: If $F_{r+1}$ contains two sets $A,B$ whose intersection $A \cap B \not \subseteq C$, for some $C \in F_{r}$, replace $A,B$ with $A \cup B$. Repeat until no two sets exist in $F_{r+1}$ whose intersection is not contained within some set $C \in F_{r}$. Knuth shows that massaging each $F_{i}$ in this manner gives a valid matroid $\mathfrak{M} = (E, \mathcal{F} = \bigcup_{i=0}^r F_i)$, satisfying the properties (C1)-(C3).

\begin{pseudo}*
  \hd{Superpose}({F_{r+1},F_r}) \\
  for $A \in F_{r+1}$ \\+
  for $B \in F_{r+1}$ \\+
  flag $\leftarrow$ \texttt{true} \\
  for $C \in F_r$ \\+
  if $A \cap B \subseteq C$ \\+
  flag $\leftarrow$ \texttt{false} \\--
  \\
  if flag = \texttt{true} \\+
  $F_{r+1} \leftarrow F_{r+1} \setminus \{A, B \}$ \\
  $F_{r+1} \leftarrow F_{r+1} \cup \{A \cup B \}$ \\




\end{pseudo}

At the end of each iteration, $r$ is incremented by one, unless $E$ is found to be a closed set in $F_{r+1}$, at which point the algorithm terminates. As described above, $E$ is the closure of the bases of the matroid, so upon reaching this point there is no point in going further: $r+1$ is the rank of the matroid.

My Julia implementation of \pr{Knuth-Matroid} can be found in THE APPENDIX. The algorithm as described produces a matroid $\mathfrak{M} = (E, \mathcal{F})$ expressed in terms of its closed sets, however, for it to be applicable to the study of fair allocation, we need several other properties of the matroid.



\section*{Finding the other properties of $\mathfrak{M}$}
Given a \pr{Knuth-Matroid}$(E, \mathrm{X}) = \mathfrak{M} = (E, \mathrm{F})$, where $\mathrm{F}$ is the list $[F_0, F_1, \ldots, F_r)$ such that $F_0 \cup F_1 \cup \ldots \cup F_r = \mathcal{F}$, the family of closed sets of $\mathfrak{M}$, and $r$ is the rank of the matroid, we want to find
\begin{enumerate}
  \item the bases $\mathcal{B}$ of $\mathfrak{M}$,
  \item the independent sets $\mathcal{I}$ of $\mathfrak{M}$,
  \item the closure function $\text{cl} : 2^E \to \mathcal{F}$,
  \item the rank function $\text{rank} : 2^E \to \mathbb{Z}^*$ of $\mathfrak{M}$, and
  \item the circuits $\mathcal{C}$ of $\mathfrak{M}$.
\end{enumerate}

Given all these, we will have arrived at a potentially useful library for interacting with matroids in the context of fair allocation.

We begin with the rank function. Since we have access to the closed sets of $\mathfrak{M}$ grouped by rank, ($\mathrm{F} = [F_0, \ldots, F_r)$), we can check the rank of an arbitrary set $S \subseteq E$ by finding the lowest $j$ such that there exists a $B \in F_j$ where $S \subseteq B$.

\begin{pseudo}*
  \hd{Rank}({S, \mathfrak{M} = (E,\mathrm{F})})\\
  for $i$ in $0 \dts \mathrm{F}.length$ do \\+
  for $B \in F[i]$ do \\+
  if $S \subseteq B$ do \\+
  return $i$ \ct{or $F[i]$ to get $\text{cl}(S)$}
\end{pseudo}

We can get implement the closure operator for free by simply returning the found set $F[i]$ instead.

Now, let's consider how we can express our matroid as $\mathfrak{M} = (E, \mathcal{B})$, where $\mathcal{B}$ is the set of bases (maximal independent sets) of $\mathfrak{M}$. Recall that a closed set is maximal for its rank, and an independent set is minimal for its rank. (C5) tells us that $E$ is the closure of all bases in $\mathfrak{M}$. We know that (B1) all bases have the same size, and since a basis is an independent set, we know that the rank of a basis equals its size. The rank of the matroid is the rank of a basis in the matroid. Hence, $\mathcal{B} \subseteq \{ A\subseteq E: |A| = r\}$, that is, the set of all $r$-sized subsets of $E$ contains $\mathcal{B}$. In the case of the uniform matroid, this would be an equality and we would be done. But in the general case, since Knuth's algorithm might have enlarged the closed sets at some ranks, certain $r$-sized subsets of $E$ are in fact dependent sets of a lower rank. To find the bases, we need to find the $r$-sized subsets of $E$ whose rank is $r$.

\begin{pseudo}*
  \hd{Bases}({\mathfrak{M} = (E, \mathrm{F})}) \\
  r $\leftarrow \mathrm{F}.length - 1$ \\
  return $\{ B \subseteq E : |B| = \pr{Rank}(B, \mathfrak{M}) = r \}$
\end{pseudo}

\section*{Results}
Knuth generates random matroids by applying random ``coarsening'' in the \pr{Enlarge} step. The randomized version, \pr{Random-Knuth-Matroid}, works by applying \pr{Superpose} immediately after generating covers, then choosing a random member $A$ of $\mathcal{F}_{r+1}$ and a random element $a \in E \setminus A$, replacing $A$ with $A \cup \{a\}$ and finally reapplying \pr{Superpose}. Table~\ref{tab:coarsening} below shows the average values obtained for various settings of the parameters. The sequence $(p_1, p_2, \ldots)$ holds the number of such coarsening steps to be applied at each iteration of the algorithm.

\begin{table*}[ht]
  \centering
  \caption{Observed mean values for \pr{Random-Knuth-Matroid}.}
  \label{tab:coarsening}
  \begin{threeparttable}
    \begin{tabular}{llllllllllll}
      \toprule
      $n$ & $(p_1, p_2, \ldots)$ & Trials                  & Bases & $|F_2|$ & $|F_3|$ & $|F_4|$ & $|F_5|$ & $|F_6|$ & $|F_7|$ & $|F_8|$ \\
      \midrule
      10  & $(6, 0, 0)$          & 44 \textsuperscript{a}  & 100.0 & 30.3    & 1.0     &         &         &         &         &         \\
      10  & $(6, 0, 0)$          & 917 \textsuperscript{b} & 76.6  & 28.3    & 25.5    & 1.0     &         &         &         &         \\
      10  & $(6, 0, 0)$          & 39 \textsuperscript{c}  & 51.6  & 31.0    & 38.5    & 27.8    & 1.0     &         &         &         \\
      10  & $(5, 1, 0)$          & 26 \textsuperscript{a}  & 107.2 & 33.3    & 1.0     &         &         &         &         &         \\
      10  & $(5, 1, 0)$          & 935 \textsuperscript{b} & 102.6 & 32.7    & 33.0    & 1.0     &         &         &         &         \\
      10  & $(5, 1, 0)$          & 39 \textsuperscript{c}  & 53.0  & 33.0    & 44.6    & 48.0    & 1.0     &         &         &         \\
      10  & $(5, 2, 0)$          & 791 \textsuperscript{a} & 108.0 & 32.5    & 1.0     &         &         &         &         &         \\
      10  & $(5, 2, 0)$          & 201 \textsuperscript{b} & 100.0 & 32.9    & 32.6    & 1.0     &         &         &         &         \\
      10  & $(5, 2, 0)$          & 8 \textsuperscript{c}   & 24.6  & 30.1    & 39.9    & 66.0    & 1.0     &         &         &         \\
      10  & $(6, 1, 0)$          & 862 \textsuperscript{a} & 99.2  & 28.4    & 1.0     &         &         &         &         &         \\
      10  & $(6, 1, 0)$          & 137 \textsuperscript{b} & 69.8  & 28.1    & 29.1    & 1.0     &         &         &         &         \\
      10  & $(6, 1, 0)$          & 1 \textsuperscript{c}   & 48.0  & 33.0    & 41.0    & 33.0    & 1.0     &         &         &         \\
      10  & $(4, 2, 0)$          & 12 \textsuperscript{a}  & 111.1 & 36.3    & 1.0     &         &         &         &         &         \\
      10  & $(4, 2, 0)$          & 950 \textsuperscript{b} & 119.2 & 35.9    & 42.5    & 1.0     &         &         &         &         \\
      10  & $(4, 2, 0)$          & 38 \textsuperscript{c}  & 73.4  & 36.4    & 52.6    & 39.4    & 1.0     &         &         &         \\
      10  & $(3, 3, 0)$          & 4 \textsuperscript{a}   & 115.0 & 39.0    & 1.0     &         &         &         &         &         \\
      10  & $(3, 3, 0)$          & 911 \textsuperscript{b} & 138.0 & 38.5    & 53.3    & 1.0     &         &         &         &         \\
      10  & $(3, 3, 0)$          & 85 \textsuperscript{c}  & 90.6  & 38.7    & 61.9    & 36.2    & 1.0     &         &         &         \\
      10  & $(0, 6, 0)$          & 767 \textsuperscript{b} & 171.8 & 45.0    & 85.6    & 1.0     &         &         &         &         \\
      10  & $(0, 6, 0)$          & 230 \textsuperscript{c} & 128.4 & 45.0    & 95.8    & 72.7    & 1.0     &         &         &         \\
      10  & $(0, 6, 0)$          & 3 \textsuperscript{d}   & 52.3  & 45.0    & 94.7    & 90.3    & 32.7    & 1.0     &         &         \\
      10  & $(0, 1, 1, 1)$       & 3 \textsuperscript{d}   & 52.3  & 45.0    & 94.7    & 90.3    & 32.7    & 1.0     &         &         \\
      \bottomrule\addlinespace[1ex]
    \end{tabular}
    \begin{tablenotes}\footnotesize
      \item [a] Averages for experiments when final rank was 3.
      \item [b] Averages for experiments when final rank was 4.
      \item [c] Averages for experiments when final rank was 5.
      \item [d] Averages for experiments when final rank was 6.
    \end{tablenotes}
  \end{threeparttable}
\end{table*}

\skelpars[2]{TODO:\ Diskutere resultatene. Sammenligne med Knuths data.}