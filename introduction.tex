\chapter{Introduction}

Fair allocation is the problem of assigning a set of items, either goods or chores, to a set of agents such that each agent is happy with what they obtain. This is an age-old problem in economics, especially for divisible items. In the divisible case, known as cake cutting, fair allocations always exist and can be computed efficiently. Fair allocation with indivisible items is a tougher nut to crack, but computer science offers a new way to approach it with algorithmic fair allocation.

Matroids are mathematical structures that are important in combinatorial optimization, and are known in computer science for being a generalization of problems for which the greedy algorithm works. In the context of fair allocation, matroids can be used to model the valuation functions of agents. When the valuation functions are matroid rank functions, efficient algorithms have been found that deliver well on fairness, efficiency and truthfulness. They can also be used to model item constraints.

This report presents an overview of fair allocation with an emphasis on the ways matroids have been applied to the field. The many different ways of defining a matroid are also discussed. Part~\ref{random-matroids} presents Knuth's classic algorithm for generating (random) matroids, and the results of running this algorithm with different input is also shown. There does not currently exist any (?) good tools for generating and using random matroids --- one goal of this project was to develop a Julia library for this purpose. A Julia implementation of the algorithms discussed can be found in the appendix. Finally, a few fair allocation algorithms for problem instances involving matroids are shortly presented and their demands of such a library discussed.